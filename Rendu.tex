Actions :

    Au début du projet les animations des différents objets étaient fortement lié aux nombres d'image par seconde. Suivant le nombre de boucle de jeu par seconde, 
la vitesse de l'animation variait. L'implémentation était tel qu'à chaque boucle de jeu chaque sprite passaient à son image suivante. Pour ralentir une animation, 
il fallait donc au minimum la diviser par 2 (changer d'image toutes les 2 boucles de jeu au lieu d'une). Pour une animation cela n'est pas encore trop grave. 
Cependant le même principe était utilisé pour le lancement des missiles. Soit on pouvait tirer un missile à chaque boucle, soit toute les 2 boucles, 3 boucles etc... 

    Cette façon d'implémenter le lancement du missile prend implicitement un paramètre qu'on ne veux pas : Le temps d'une boucle de jeu.
Même si via le code on décide de pouvoir lancer un missile toute les 2 boucles de jeu, on ne connait pas le temps d'une boucle. On ne connait donc pas le nombre de missile 
que le joueur peut lancer dans un temps donné. Pour palier à cela, nous avons crée une stucture action_t qui utilise le temps pour restreindre une "action" plutôt que d'utiliser 
les boucles de jeu. De cette manière, le jeu peut tourner à 30 FPS ou à 120 FPS, le temps entre deux missiles ne serra pas impacté.

typedef struct action {
    int cooldown; // Temps minimal pour réaliser à nouveau l'action (en ms)
    Uint32 lastTimeActionWasPerformed; // Dernière fois que l'action a été réalisé (en ms)
    int currentState; // Permet de modifier le comportement de l'action
} action_t;

    Nous utilisons cette stucture pour 3 actions : 
    - Les animations : Cela permet d'avoir une réelle influence sur la vitesse de l'animation. Quand l'oiseau decend et que l'animation est sensé ralentir, nous pouvons 
    simplement augmenté la variable "cooldown". Cela va réduire le nombre de fois où nous changeons l'image de l'oiseau dans un temps donné. De la même manière, quand on appuie sur
    la touche pour remonter, nous mettons la variable "cooldown" à 0, ce qui revient à changer l'image de l'oiseau à chaque boucle de jeu.
    - Lançer des missiles : /* TODO */
    - Faire clignoter l'oiseau : /* TODO */

    Nous avons implémenter une fonction permettant simplement de vérifier si l'action donné en paramètre est faisable ou non. Si la fonction renvoie true, on part du principe que l'action est
    faisable et on l'effectue.

int isPossibleAction(action_t* action) {
    Uint32 currentTime = SDL_GetTicks(); // On précupère le temps présent
    if((currentTime - action->lastTimeActionWasPerformed) >= action->cooldown)
    {
        // L'action est possible, on reset le cooldown de l'action
        action->lastTimeActionWasPerformed = currentTime;
        return true;
    }
    return false;
}


Timers :

typedef struct timer {
    timer_id_t id; // ID du timer récupéré via la fonction add_timer
    timer_callback_func_t callback; // Fonction que le timer va appeler
}myTimer_t;


Sinusoidales :






// Pour lancer un missile
if(espace && isPossibleAction(&object->actions[shotMissile]))
{
    if(object->state != OBJECT_STATE_DEAD)
    {
        animation_missile_add(object);
    }   
}
